
\section{Prerequisites}

There is certain terminology that one must be familiar with if we are to begin to understand Dijkstra's Algorithm, which is a graph theoretic algorithm. This section aims to make clear what terminology will be used in subsequent sections in brevity.\\

Let's review some relevant definitions.

\begin{defn}
A \vocab{graph} is a representation of a set of objects and the pairwise relationships between them. \\ \textit{Taken from Algorithms Illuminated}
\end{defn}

In simple language, a graph is a way of visualizing the relationships between different related things. For example, say we have a graph of countries, i.e. the set of objects is all of the nations on the planet and every object is a nation. In such a case, perhaps, some relationships that could be represented could be trade-relationships or, say, alliances.  \\

Considering the properties of graphs, there is specific vocabulary pertaining to one singular object among the set of objects mentioned earlier.  
\begin{defn}
A node of a graph is an object, it represents something specific and distinct among the set of objects that composes it.
\end{defn}

A node is a direct way to refer to an object among the set of objects. Those two ideas are equivalent and it just could have easily been said that a graph is a set of nodes and their relationships instead of being a set of objects and their relationships.

\begin{defn}
An edge is a physical representation of the pairwise relationships between nodes.
\end{defn}

This definition can sound daunting - let's break it down. An edge is a way to visualize the relationship between nodes. That's what "physical representation" refers to. Now, what does "pairwise" allude to? The word "pairwise" refers to the fact that edges are demarcated between node pairs, i.e. from one node to another node. \\

Here is an example of what that would look like,
% insert image of that

