
\section{Prelude}

\begin{theorem} [Relaxation of an Edge]

Relaxing an edge is the idea of reducing the distance to a node if the distance found to that node in the current iteration is smaller than the tabulated distance. \\

More formally, if the current node at the current iteration is node $u$, then, given that $d_x$ represents the tabulated distance to node $x$, upon discovering node $v$ in $u$'s adjacency list on edge of weight $w$,

\[
    \texttt{if } (d_v > d_u + w): \\
        d_v = d_u + w
\]

or, the more concise

$$d_v = min(d_v, d_u + w)$$

\end{theorem}